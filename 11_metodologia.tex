\chapter{METODOLOGIA}
Para este trabalho, foram utilizados como fonte de dados para o algorítimo, o IDS Kitsune e seus datasets com modelos comums de ataques \cite{kitsune}. A metodologia terá uma abordagem experimental, fazendo o teste em diferentes tipos de ambientes domésticos e hardwares. 

\hfill
\newcolumntype{L}{>{\RaggedRight\arraybackslash}X}
\begin{table}[!hbt]
\renewcommand\thetable{2}
\setlength\tabcolsep{3pt} 
\caption{Hardware utilizado nos testes}
\centering
 \begin{tabularx}{\textwidth}{@{}*{4}{L} @{}} 
 \toprule
 Hardware & RAM & Processamento & Sistema \\ [0.5ex]
 \midrule
 Computador Pessoal &
 16 GB &
 3.6GHz &
 Windows 10 \\ 
 \addlinespace
 Laptop Pessoal &
 8 GB &
 2.30GHz &
 Linux \\ 
 \addlinespace
 Computador Pessoal &
 16GB &
 3.5GHz &
 MacOS \\ 
 \addlinespace
 \bottomrule
\end{tabularx}
\fonte{Autoria Própria, 2022} 
\end{table}

\hfill
\newcolumntype{L}{>{\RaggedRight\arraybackslash}X}
\begin{table}[!hbt]
\renewcommand\thetable{3}
\setlength\tabcolsep{3pt} 
\caption{Dispositivos IoT conectados a Rede}
\centering
 \begin{tabularx}{\textwidth}{@{}*{4}{L} @{}} 
 \toprule
 Dispositivo & Tipo & Fabricante \\ [0.5ex]
 \midrule
 Echo Dot (3ª Geração) &
 Auto falante Inteligente &
 Amazon \\
 \addlinespace
 Camera IP &
 Câmera de Segurança &
 Intelbraz \\
 \addlinespace
 DVR &
 Gravador Digital Inteligente &
 HiLook \\
 \addlinespace
 \bottomrule
\end{tabularx}
\fonte{Autoria Própria, 2022} 
\end{table}

\section{Implantação na Rede Doméstica}
Para a implementação, serão realizados os testes com o IDS instalado diretamente no dispositivo alvo, capturando os pacotes do mesmo, também outra abordagem será a instalação direto no roteador da rede.

\section{ALGORÍTIMO DE DETECÇÃO DE ANOMALIAS}
O algorítimo será implementado em Python3 e adaptado para o uso em ambientes domésticos, utilizando-se de estratégias de análise menos custosas e automatizações, que tornarão o uso menos complexo. 

A implementação do algorítimo e adaptação se dará pela captura automática de pacotes em quantidades baixas (20 mil por análise), para assim tornar o processamento desses pacotes mais rápido. Também, diferentemente do funcionamento do Kitsune, o método de captura, análise e processamento serão de forma automática e recursiva, tornando o processo.

\section{Processamento dos pacotes}
O processamento dos pacotes possui três fases, a primeira captura os pacotes a partir de bibliotecas externas, na segunda fase, o algorítimo gera uma instancia do pacote, com informações como: ip de origem, ip de destino, protocolos, tempo de resposta, etc, e por fim na terceira fase é onde a Rede Neural Artificial Analisa o comportamento de cada instancia.

\section{Exclusão dos arquivos TCPDUMP}
Os arquivos tcpdump gerados, contém informações de pacotes de rede usados durante o processamento dos dados, por isso após seu uso, são excluidos do armazenamento, assim impedindo o acesso não autorizado.

\chapter{CRONOGRAMA}

O desenvolvimento deste trabalho se dará da seguinte forma:

\begin{enumerate}
	\item \label{ela-pro} Elaboração da proposta.
	\item \label{anI} Análise dos métodos de ataques a dispositivos IoT.
	\item \label{anII} Configurações dos ambientes de testes.
	\item \label{anIII} Implementação do algorítimo nos ambientes de teste. 
	\item \label{dI} Análise dos modelos implementados em hardware estudados no item IV.
		\item \label{dII} Validação dos resultados preliminares de desempenho.
	\item \label{tec} Teste e correções.
		\subitem Comparar desempenho entre diferentes ambientes.
	\item \label{esc-tcII} Escrita do TCC II.
\end{enumerate}

\begin{table}[!htbp]
	\centering
		\setlength\tabcolsep{15pt}\begin{tabular*}{\textwidth}{|c @{\extracolsep{\fill}} |c|c|c|c|c|c|c|}
		\hline
		& \multicolumn{5}{c|}{2022}&\multicolumn{2}{c|}{2023} \\
		\hline
		\hspace{-0.4cm} Atvidades \hspace{0.1cm}  & AGO & SET & OUT & NOV & DEZ & JAN & FEV \\
		\hline
		\hspace{-0.3cm}\ref{ela-pro} & X & & & & & & \\
		\hline
		\hspace{-0.3cm}\ref{anI} & & X & X & & & & \\
		\hline	
		\hspace{-0.3cm}\ref{anII} & & & X & & & & \\
		\hline			
		\hspace{-0.3cm}\ref{anIII} & & & X & X & & & \\
		\hline	
		\hspace{-0.3cm}\ref{dI} & & & & & X & & \\
		\hline
		\hspace{-0.3cm}\ref{dII} & & & & & & X & \\
		\hline	
		\hspace{-0.3cm}\ref{tec} & & & & & & & X \\
		\hline	
		\hspace{-0.3cm}\ref{esc-tcII} & & & & & & & X \\
		\hline	
		\end{tabular*}
\end{table}