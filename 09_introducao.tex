\chapter{INTRODUÇÃO}
Dado o crescimento exponencial do uso da internet, em virtude da convergência de áreas para dentro da computação, aumenta-se a complexidade e a necessidade de se manter cuidadoso diante de problemas de segurança. \cite{mendes2020redes}

Antes, somente computadores e celulares eram conectados a internet, mas com a chegada dos dispositivos IoT (Internet of Things), permitiu-se que qualquer objeto seja conectado à rede, o que também abriu brechas para ainda mais problemas de segurança \cite{alrawi}.

 A IoT é uma rede de objetos físicos, que inclui desde utensílios domésticos como lâmpadas, veículos, sensores, câmeras de vigilanciana e outros objetos conectados com a Internet. No ano atual, existem cerca de 12.2 bilhões de dispositivos IoT ativos e espera-se que, até 2025, haverá aproximadamente 27 bilhões. \cite{Gomes_2022}; \cite{IoTResearch}.

Devido a este crescimento, necessita-se também de técnicas mais eficazes para garantir a integridade e segurança dos dados pessoais, visto que dispositivos IoT domésticos acabam sendo um alvo fácil devido sua baixa segurança \cite{Otoum}. Uma das técnicas de combater estes ataques é com o uso de IDSs.

Sistemas de Detecção de Intrusão (IDS), fazem a detecção de possíveis ameaças a partir da classificação de padrões ou comportamento do tráfego de rede \cite{Shurman}. Kitsune é um algorítimo baseado em Redes Neurais Artificiais não supervisionadas, open-source, capaz de aprender padrões e comportamentos complexos para diferenciar um trafego normal ou um ataque. \cite{kitsune}

A ideia de usar o monitoramento de pacotes de dados foi empregado pela primeira vez por James Anderson em 1980. A partir deste momento, pesquisadores desenvolveram vários métodos para melhorar o desempenho e precisão das analises. Uma opção é unir a capacidade de uma rede neural em processar padrões ao monitoramento de dados.

Pretende-se usar como base o algorítimo Kitsune \cite{kitsune}, e realizar uma pesquisa de aplicação e viabilidade do uso de Redes Neurais Artificiais para detectar intrusões em redes domésticas.