% Isto é um exemplo de Ficha Catalográfica, ou ``Dados internacionais de
% catalogação-na-publicação''. Você pode utilizar este modelo como referência. 
% Porém, a biblioteca da universidade lhe fornecerá um PDF com a ficha catalográfica definitiva após a defesa do trabalho. Quando estiver com o documento, salve-o como PDF na pasta pdf do seu projeto e substitua todo o conteúdo de implementação deste arquivo pelo comando abaixo:

% \begin{fichacatalografica}
%     \includepdf{pdf/ficha_catalografica.pdf}
% \end{fichacatalografica}

% Comente ou retire todas as linhas abaixo após inserir o documento da ficha catalográfica

\begin{fichacatalografica}
	\sffamily
	\vspace*{2in}					% Posição vertical
    
	\begin{center}					% Minipage Centralizado
    	\fbox{
    	    \begin{minipage}[c][8cm]{13.5cm}		% Largura
            	\small
            	\imprimirautor
            	%Sobrenome, Nome do autor
            	
            	\hspace{0.5cm} \imprimirtitulo  / \imprimirautor. --
            	\cidade, \estado - 
            	
            	\hspace{0.5cm} \thelastpage p. : il. (algumas color.) ; 30 cm.\\
            	
            	\hspace{0.5cm} \genorientador~\orientador\\
            	
            	\hspace{0.5cm}
            	\parbox[t]{\textwidth}{Trabalho de Conclusão de Curso~--~IFMA - Caxias/MA,
            	\Month de \the\year.}\\ 
            	
            	\hspace{0.5cm}
            		1. Palavra-chave1.
            		2. Palavra-chave2.
            		2. Palavra-chave3.
            		I. Orientador.
            		II. Universidade xxx.
            		III. Faculdade de xxx.
            		IV. Título 			
        	\end{minipage}
    	}
	\end{center}

\end{fichacatalografica}
